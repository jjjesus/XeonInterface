\documentclass[12pt]{article}

\usepackage{xcolor}
\usepackage{float}
\usepackage{graphicx}
\usepackage{fancyhdr}

\usepackage[utf8]{inputenc}
\usepackage{setspace}

\usepackage[dvips,letterpaper,margin=1in]{geometry}

\setlength{\parindent}{0pt}
\setlength{\headheight}{15pt} % fancyhdr wants at least 14.5pt

% Header
\pagestyle{fancy}
\lhead{MFRFS K3 Xeon-D Interface Description}
\rhead{July 28, 2020}
\renewcommand{\headrulewidth}{0.4pt}
\renewcommand{\footrulewidth}{0.4pt}

% Start document
\begin{document}

%%%%%%%%%%%%%%%%%%%% Title Page %%%%%%%%%%%%%%%%%%%%%%%%
\thispagestyle{empty}
\begin{titlepage}
\begin{center}
        \vspace*{1cm}

        \LARGE{Multi-Function Radio Frequency System (MFRFS) K3 \\
            Xeon-D Interface Description}

        \vspace{0.5cm}
        \LARGE
        % Subtitle

        \vspace{1.5cm}

        \normalsize

        John Jesus \\
        July 28, 2020

        \vfill



        \vspace{0.8cm}




\end{center}
\end{titlepage}

\tableofcontents
\newpage

%%%%%%%%%%%%%%%%% Start of Body %%%%%%%%%%%%%%%%%%%%%%%%
\section{Scope}
\subsection{Purpose}
This document is describes interfaces to the Xeon-D SoC of
the XES Xpedite7674 Single Board Computer as planned for MFRFS K3.

\subsection{References}
\begin{enumerate}
    \item XPedite7674 Users Manual Revision C (Extreme Engineering Solutions, February 1, 2018) \label{ref:board_man}
    \item XIt1088 User’s Manual Revision A (Extreme Engineering Solutions, May 29, 2019) \label{ref:rtm_man}
    \item XPEDITE7674 Schematic Diagram SCH90030490 Revision C (Extreme Engineering Solutions, February 28, 2020) \label{ref:schematic}
    \item Mercury Systems TRRUST-Stor SSD MSD256/512 and MDR256/512 (Mercury Systems, July 25, 2016) \label{ref:mercurty_ssd}
\end{enumerate}

\section{Interfaces}

Interfaces are depicted in Figure \ref{fig:inteface}.  Details of each
interface are provided in subsections below.

\begin{figure}[H]
\begin{center}
\includegraphics[width=1.0\textwidth]{img/interface}
\caption{Xeon-D Interfaces on the XPedite7674 for MFRFS K3}
\label{fig:inteface}
\end{center}
\end{figure}


% %%%%%%%%%%%%%%%%%%%%%%%%%%%%%%%%%%%%%%%%%%%%%%%
% sata (boot flash)
% %%%%%%%%%%%%%%%%%%%%%%%%%%%%%%%%%%%%%%%%%%%%%%%

\subsection{SATA (local boot flash)}
\label{section:sata}

This SATA interface is to the Linux boot image and initial ram disk.
The K3 application does not use this interface.
The only tie between the Linux boot image and the K3 application is 
an entry in the initial ram disk in \texttt{/etc/services.d} that
starts a K3 executable called \texttt{CSWMain}.
\texttt{CSWMain} is a shim program that loads configuration files
and libraries from the External SSD (Section \ref{section:sata2}) and
launches them.


% %%%%%%%%%%%%%%%%%%%%%%%%%%%%%%%%%%%%%%%%%%%%%%%
% sata2 (external ssd)
% %%%%%%%%%%%%%%%%%%%%%%%%%%%%%%%%%%%%%%%%%%%%%%%

\subsection{SATA (External SSD)}
\label{section:sata2}

An External Solid-State Drive (SSD) is the Mercury Systems TRRUST-Stor SSD MSD512 Self-Encrypted Drive mounted in a 3-U carrier in Slot 8 of the VPX chassis.  The Xeon-D has a SATA connection to this SSD through the VPX backplane and uses the serial port interface through the SCIM (Section \ref{section:sata2}) to send key information to unlock the SSD. 
%%%
\vspace{0.8cm}

\textcolor{red}{\textbf{Question for Nick:} Do you think we need to
continue using a Self-Encrypted Drive like the Mercury TRRUST-Stor?}

%%%

\end{document}

